\documentclass[11pt,titlepage]{article}

\input{art/cozydefs.tex}

\usepackage{listings}
\usepackage{import}
\usepackage{epsfig}
\usepackage{verbatim}

\newcommand{\function}[1]{\lstinline{#1}()}
\newcommand{\datatype}[1]{\lstinline{#1}}
% TODO fix this
%\newcommand{\function}[1]{\verb+$#1$+}
%\newcommand{\function}[1]{\verb+#1()+}
\newcommand{\module}[1]{\texttt{#1}}


\begin{document}
\begin{titlepage}
%\maketitle
\lstset{language=c}
\newcommand\HRule{\noindent\rule{\linewidth}{1.5pt}}
\vspace*{\stretch{1}}
\HRule
\begin{center}
	\LARGE cozybit Embedded Zeroconf Module
\end{center}
\HRule
\vspace*{\stretch{2}}
\vspace{1in}
\begin{center}
	\Large
% nothing here
{\small January $21^{st}$, 2007}\\
%{\small \date}\\
\rule{0pt}{20ex}\\
\includegraphics*{art/cozybit-logo.eps}\\
\vspace{3ex}
{\scriptsize 
\begin{tabular}{l l}
Tel: +1 415 974 6770 	& 165 Jessie St. \\
Fax: +1 415 974 6771    & San Francisco, CA 94105 \\
http://www.cozybit.com  & USA \\
\end{tabular}}
\end{center}
\end{titlepage}
\clearpage

%%%%%%%%%%%%%%%%%%%%%%%%%%%%%%%%%%%%%%%%%%%%%%%%%%%%%%%%%%%%%%%%%%%%%%%%%%%%%%%

\section{cozybit Zeroconf module}

The cozybit Zeroconf module allows devices to advertise services and announce 
themselves to users without requiring extensive network configuration.  With 
Zeroconf, an embedded system can:

\begin{itemize}
	\item Quickly point users to its web-based configuration page.
	\item Announce a wireless VOIP handset on the network.
	\item Enable users to browse for services such as printers, media
		  devices, and network-attached storage.
\end{itemize}

\subsection{Features}

The cozybit Zeroconf module provides:

\begin{itemize}
	\item Multicast DNS (mDNS) for automatic name resolution without the
		  presence of a DNS server.
	\item mDNS Responder, enabling the device to advertise its services on
		  to DNS-SD (Service Discovery) browsers on the network.
	\item mDNS Responder API, allowing the developer to configure the device
		  to advertise a desired name and service or services.
	\item IP4LL (Link Local) address self-assignment when it is needed.
	\item Adherence to current Zeroconf and mDNS draft standards and 
		  compatibility with Zeroconf implementations on Apple, Windows, and
		  Linux PCs.
	\item OS and TCP/IP stack independence.
\end{itemize}

\subsection{Details}

The cozybit Zeroconf module is designed for small embedded systems.  The mDNS 
Responder runs in a single thread and does not require the use of dynamically 
allocated memory.  The mDNS service, at multicast address 224.0.0.251, uses 
UDP port 5353.

IP4LL allows devices to self-assign IP addresses in the 169.254/16 prefix. 
This allows devices to communicate on networks without DHCP or user 
configuration and is especially useful in ad-hock wireless networks.  The mDNS
Responder is capable of using an assigned IP or an IP4LL-provided address when
needed.

The module consists of an mDNS Responder thread and an optional IP4LL thread 
for Link Local address self-assignment.  An OS abstraction layer allows the 
module to be easily ported to many platforms.

\begin{center}
\psfig{figure=./figures/arch.eps,angle=0,width=40ex}
\end{center}

At this time, the cozybit Zeroconf module has been ported to Linux and the 
Marvell 8388V standalone SDK.  The module's footprint when compiled with the
ARM ADS toolchain is approximately 2KB.

\subsection{Standards}

Please refer to the following draft standards and RFCs for details about the
Zeroconf draft standard:

\vspace{1ex}
\begin{tabular}{|l|l|}
\hline
mDNS & http://files.multicastdns.org/draft-cheshire-dnsext-multicastdns.txt \\
\hline
DNS-SD & http://files.dns-sd.org/draft-cheshire-dnsext-dns-sd.txt \\
\hline
DNS SRV & http://www.faqs.org/rfcs/rfc2782.html \\
\hline
IPV4LL & http://files.zeroconf.org/rfc3927.txt \\
\hline
\end{tabular}

\end{document}
